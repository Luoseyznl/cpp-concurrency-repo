\documentclass[tikz,border=10pt]{standalone}
\usepackage{tikz}
\usetikzlibrary{shapes.multipart, shadows, positioning, arrows.meta, calc, backgrounds, decorations.pathreplacing}
\usepackage{newtxtext,newtxmath}

\definecolor{cFlagDraw}{RGB}{97, 97, 97}
\definecolor{cFlagFill}{RGB}{245, 245, 245}
\definecolor{cFlagText}{RGB}{66, 66, 66}

\definecolor{cGenDraw}{RGB}{21, 101, 192}
\definecolor{cGenFill}{RGB}{227, 242, 253}
\definecolor{cGenText}{RGB}{13, 71, 161}

\definecolor{cSpecDraw}{RGB}{46, 125, 50}
\definecolor{cSpecFill}{RGB}{232, 245, 233}
\definecolor{cSpecText}{RGB}{27, 94, 32}

\begin{document}
\begin{tikzpicture}[
		font=\footnotesize\sffamily,
		% 定义 UML 类图风格的节点
		classNode/.style={
				rectangle split,
				rectangle split parts=2,
				draw=#1Draw,
				fill=#1Fill,
				thick,
				rounded corners=3pt,
				align=left,
				text width=5.5cm,
				drop shadow={opacity=0.15, shadow xshift=2pt, shadow yshift=-2pt},
				rectangle split part fill={#1Fill, white} % 标题有背景,内容白色
			},
		% 连线样式
		arrowConn/.style={
				draw=black!40,
				line width=1.5pt,
				{Latex[width=3mm, length=3mm]}-, % 箭头指向下方,表示"基于"
			}
	]

	% --- Top: 数值/指针特化 ---
	\node[classNode=cSpec] (top) {
		\textbf{std::atomic$<$Integral / T*$ >$}
		\nodepart{second}
		\texttt{+ fetch\_add / fetch\_sub}\\
		\texttt{+ fetch\_and / \_or / \_xor} (Integral)\\
		\texttt{+ operator++, --, +=, -=}
	};

	% --- Middle: 通用类型 ---
	\node[classNode=cGen, below=0.6cm of top] (mid) {
		\textbf{std::atomic$<$T / bool$>$}
		\nodepart{second}
		\texttt{+ load() / store()}\\
		\texttt{+ exchange()}\\
		\texttt{+ compare\_exchange\_weak/strong}\\
		\texttt{+ wait() / notify\_*} (C++20)
	};

	% --- Bottom: 标志位 ---
	\node[classNode=cFlag, below=0.6cm of mid] (bot) {
		\textbf{std::atomic\_flag}
		\nodepart{second}
		\texttt{+ test\_and\_set()}\\
		\texttt{+ clear()}\\
		\textit{*唯一的无锁保证 (Lock-free Guarantee)}
	};

	% 连接线:空心宽箭头,表示"扩展"
	\draw[arrowConn] (mid.north) -- (top.south);
	\draw[arrowConn] (bot.north) -- (mid.south);

	% 左侧的大括号
	\draw[decorate, decoration={brace, amplitude=10pt, raise=5pt}, thick, color=black!60]
	(bot.south west) -- (top.north west)
	node [midway, xshift=-3.5em, align=center, rotate=90, font=\bfseries\small, color=black!70]
	{Capability Hierarchy};

	% 右侧的注解
	\node[anchor=west, text=cSpecText, font=\scriptsize] at (top.east)
	{$\leftarrow$ 支持算术运算};

	\node[anchor=west, text=cGenText, font=\scriptsize] at (mid.east)
	{$\leftarrow$ 支持 CAS 与内存序};

	\node[anchor=west, text=cFlagText, font=\scriptsize] at (bot.east)
	{$\leftarrow$ 纯硬件指令封装};

\end{tikzpicture}
\end{document}